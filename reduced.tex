\documentclass{article}
\usepackage{amsmath}
\usepackage{amsthm}
\usepackage{amssymb}
\usepackage{blindtext}
\usepackage{longtable}
\usepackage{titlesec}
\usepackage{booktabs}
\usepackage{xcolor}
\usepackage{fancyhdr}
\usepackage{multirow}
\usepackage{paralist}
\usepackage[margin=0.5in]{geometry}
\vspace{1mm} %5mm vertical space
\usepackage{titlesec}
\title{DS Cheat Sheet}
\author{Anirudh S. Kumar, 2021517, CSAI}


% We define two structures: one for theorems and another one for definitions
% since we used \newtheorem* (with a star) for definitions, the
% definitions won't be numbered
\newtheorem{theo}{Theorem}[section]
\theoremstyle{definition}
\newtheorem*{defi}{Definition}
\newtheorem{prop}[section]{Proposition}
\theoremstyle{definition}
\newtheorem{corollary}{Corollary}[theo] % Use theorem counter as `parent`


\newtheorem{manualpropinner}{Proposition}
\newenvironment{manualprop}[1]{%
  \renewcommand\themanualpropinner{#1}%
  \manualpropinner
}{\endmanualpropinner}



\newtheorem{manualtheoreminner}{Theorem}
\newenvironment{manualtheorem}[1]{%
  \renewcommand\themanualtheoreminner{#1}%
  \manualtheoreminner
}{\endmanualtheoreminner}


\newtheorem{manualcoroinner}{Corollary}
\newenvironment{manualcoro}[1]{%
  \renewcommand\themanualcoroinner{#1}%
  \manualcoroinner
}{\endmanualcoroinner}


\newtheorem{manuallemmainner}{Lemma}
\newenvironment{manuallemma}[1]{%
  \renewcommand\themanuallemmainner{#1}%
  \manuallemmainner
}{\endmanuallemmainner}

\newtheoremstyle{named}{}{}{\itshape}{}{\bfseries}{.}{.5em}{\thmnote{#3's }#1}
\theoremstyle{named}
\newtheorem*{namedtheorem}{Theorem}

\begin{document}
\maketitle
% \tableofcontents
\pagestyle{fancy}
\fancyhf{}
\renewcommand{\headrulewidth}{0pt}
\fancyfoot[R]{Anirudh S. Kumar, 2021517}
\fancyfoot[C]{\thepage}

\section{\underline{Relations and Functions}}

\begin{defi}[Principle of Mathematical Induction(PMI]
Let $X \subseteq \mathbb{N}$, and suppose the following conditions hold for X:
\begin{enumerate}
    \item $0 \in X$
    \item If $n \in X$, then $n+1 \in X$ also
\end{enumerate}
then $X = \mathbb{N}$
\end{defi}


\begin{defi}[Principle of Mathematical Induction - Strong Version (SPMI)]
Let $X \subseteq \mathbb{N}$, and suppose the following conditions hold for X:
\begin{enumerate}
    \item $0 \in X$
    \item If $0,1,.., n \in X$, then $n+1 \in X$ also
\end{enumerate}
then $X = \mathbb{N}$
\end{defi}

\begin{defi}[Well-Ordering Priciple(WOP)]
If $X$ is a non-empty subset of $\mathbb{N}$, then $X$ possesses a smallest or least element. 
More precisely, $$\exists \alpha \in X, \alpha \leq x , \forall x \in X$$ 
\end{defi}

\begin{manualprop}{1}
the following three statements are equivalent
\begin{itemize}
\item PMI
\item SPMI
\item WOP
\end{itemize}
\end{manualprop}

\begin{manualprop}{2}
Let $f: X \xrightarrow[]{} Y$ and $g: Y \xrightarrow[]{} Z$. Then
\begin{itemize}
    \item If $f$ and $g$ are injective, then $g \circ f$ is also injective
    \item If $f$ and $g$ are surjective, then $g \circ f$ is also surjective
    \item If $f$ and $g$ are bijective, then $g \circ f$ is also bijective
    \item For any function $f: X \xrightarrow[]{} Y, \exists$ a set $Z$, an injective function $h: Z \xrightarrow[]{} Y$, and a surjective function $g:X \xrightarrow[]{} Z$ such that $f = g \circ h$.
\end{itemize}
In other words, every function can be decomposed as the composition of a surjective function followed by an injective function
\end{manualprop}

\begin{defi}
A partition of a set $X$ is a family of non-empty subsets of X, which are pair-wise disjoint and whose union is all of $X$. The subsets in a partition are called the parts of the partition.
\end{defi}

\begin{manualprop}{3}
If $R$ is a equivalence relation on $X$, then $R$ induces a \textbf{\emph{partition}} of $X$, the parts of the partition being the\textbf{\emph{equivalence classes}}, i.e. the equivalence classes are pair-wise disjoint subsets whose union is the whole set $X$. Conversely, given any partition of a set $X$, there exists a corresponding equivalence relation $R$ on $X$.
\end{manualprop}

\begin{defi}
Given any set $X$, the relation $\Delta_{X} = \{(x, x): x \in X\}$ is the smallest possible reflexive relation on $X$ - it is called the \emph{diagonal} on set $X$
\end{defi}

\section{\underline{Posets}}
\subsection{\underline{Introduction and Isomorphism}}

\begin{defi}
A relation $R$ is said to be \textbf{{antisymmetric}} if whenever $xRy$ and $yRx$ both hold, then $x = y$.
Alternatively, R is \textbf{{antisymmetric}} if $R \cap R^{-1} \subseteq \Delta_{X}$ \newline
A relation $R$ on a set $X$ is said to be a partial ordering if it is \emph{reflexive, antisymmetric} and \emph{transitive}.  
\newline
A pair $<X, \leq>$, where $\leq$ is a partial ordering on the set $X$ is called a \textbf{poset}
\end{defi}

\begin{defi}
Let $<X, \leq >$ be a poset. We say that an element $x \in X$ is an \textbf{immediate predecessor} of an element $y \in X$ if $x < y$ but there is no $t \in X$ such that $x < t < y$. Symbol - $\Diamond$ 
\end{defi}

\begin{manualprop}{4}
Let $<X, \leq >$ be a \textbf{finite} poset. Then for any $x, y \in X, x < y$ holds \textbf{iff} there exists elements $x_1, x_2, ... x_k \in X$ such that $x \Diamond x_1  \Diamond x_2  \Diamond  .....  \Diamond x_k  \Diamond y$ , $k\geq 0$
\end{manualprop}

\begin{defi}
Let $<X, \leq >$ be a poset. Then an element $a \in X$ is said to be a \textbf{minimal} element if there is no element $x \in X$ such that $x < a$. Similarly, an element $a \in X$ is said to be a maximal if there is no $x \in X$ such that  $a < x$
\end{defi}

\begin{manualprop}{5}
Every finite poset $<X, \leq>$ has at least one minimal element.
\end{manualprop}

\begin{defi}
An element $a$ of a $<X, \leq>$ is said to be a least or minimum element if $a \leq x$ holds  $\forall x \in X$, and if it exists, must be unique.
\end{defi}

\begin{manualprop}{6}
Every partial ordering on a finite set $X$ has a \textbf{linear extension}. In other words, if $<X, \leq>$ is a finite poset, then there exists a linear (total) ordering "$\#$" on $X$ such that $\forall x, y \in X$, $x \leq y \implies x \# y$
\end{manualprop}

\begin{defi}
Two posets $<X, R>$ and $<Y, S>$ are said to be \textbf{isomorphic} if there exists a bijection $f: X \xrightarrow[]{} Y$ (called an \textbf{isomorphism}) such that $xRz$ iff $f(x)Sf(z)$ $\forall x, y \in X$ 
\end{defi}

\begin{defi}
If $<X, R>$ and $<Y, S>$ are posets, a mapping $f: X \xrightarrow[]{} Y$ is said to be an \textbf{embedding} if 
\begin{enumerate}
\item $f$ is injective and 
\item $f(x)Sf(z)$ iff $xRz$ $\forall x, z \in X$
\end{enumerate}
\end{defi}

\begin{manualprop}{7}
For every poset $<X, R>$, there exists an embedding into the poset $<\mathbb{P}(X), \subseteq>$
\end{manualprop}
Let $f: X \xrightarrow[]{} \mathbb{P}(x)$ and 
$ f(x) = \{z \in X: z \leq x\} $

\subsection{\underline{Chains and Anti Chains}}
\begin{defi}
Let P be a fixed but arbitrary poset $<X, \leq>$. Elements $x, y \in X$ are said to be \textbf{comparable} if either $x \leq y$ or $ y \leq X$; else, they are said to be \textbf{incomparable}
\newline
A set $A \subseteq X$ is said to be independent or an antichain in P if the elements of A are mutually \textcolor{red}{incomparable}. A set $A \subseteq X$ is said to be a chain in P if the elements of A are mutually \textcolor{red}{comparable}.
\end{defi}

\begin{defi}
The \textbf{independence number} of P, denoted by $\alpha(P)$ is the size of the largest antichain in P; \newline i.e.  $\alpha(P) = max\{|A|$: A is an antichain in $P$\}.
\newline
Similarly we wil use $\omega(P)$ for the size of the largest chain in $P$.
\end{defi}

\begin{manualprop}{8}[important]
For every finite poset $P = <X, \leq>$, we have $\alpha(P)\omega(P) \geq |X|$
\end{manualprop}

\begin{namedtheorem}[Erdős-Szekeres] 
An arbitrarily finite real sequence $<x_k>$ of length at least $n^2 + 1$ contains a monotone subsequence of length at least $n$
\end{namedtheorem}

\begin{manualcoro}{1.1}
An infinite real sequence $<x_k>$ contains a monotone subsequence of any arbitrary finite length
\end{manualcoro}

\begin{namedtheorem}[Dilworth Decomposition]
Let $P = <X, \le>$ be a finite poset. Then $X$ can be decomposed into the disjoint union of $\alpha(P)$ chains, and cannot be decomposed into any smaller number of chains. i.e. we get a \textbf{partition} of $X$ into $\alpha(P)$ chains, which is the best possible 
\end{namedtheorem}

\subsection{\underline{Lattices}}

\begin{defi}

\begin{itemize}
\item Let $P = <X, \leq>$ be a poset, and let $A$ be a non-empty subset of $X$. An element $u \in X$ is said to be an \textbf{upper bound} for $A$ if $x \leq u$ $\forall x \in A$.
\item $u$ need not be an element of $A$. 
\item An element $u\in X$ is said to be a \textbf{least upper bound}(lub) or \textbf{supremum}(sup) for $A$ if $u$ is an upper bound for $A$.
\item $sup(A)$, if it exists at all, has to be unique
\end{itemize}
\end{defi}

We can define \textbf{lower bounds}, and the \textbf{greatest lower bound}(glb) or \textbf{infimum}(inf) for any non-empty subset $A$ of $X$ 


\begin{defi}
A poset $<X, \leq>$ in which every \textit{pair} of elements has both a supremum and an infimum is called a \textbf{lattice}. 
\end{defi}

\begin{manualprop}{9}
For any $a,b,c,d$ in a lattice $<X, \leq>$, we have
\begin{enumerate}
\item $a \leq a \vee b $  and $a \wedge b \leq a$
\item If $a \leq b$ and $c \leq d$, then $a \vee c \leq b \vee d $ and $a \wedge c \leq b \wedge d$
\end{enumerate}
\end{manualprop}

\begin{defi}
Dual of a a lattice $<X, R>$ is $<X, R^{-1}>$. Statements about join in the original lattice become statements about meet in the dual lattice,and vice-versa.
\end{defi}

\begin{defi}
A lattice $L = <X, \leq>$ is said to be \textbf{distributive} if 
  
\begin{enumerate}
    \item   $a \vee (b \wedge c) = (a \vee b) \wedge (a \vee c)$
    \item   $a \wedge (b \vee c) = (a \wedge b) \vee (a \wedge c)$
\end{enumerate}

\end{defi}
\begin{manualprop}{11}
If either of the two above statements holds, then the other one also holds
\end{manualprop}

\begin{manualprop}{12}
Let $L = <X, \leq>$ be a lattice with 0 and 1. Then, the following holds $\forall a \in X$
\begin{itemize}
    \item $a \vee 1 = 1$ and $ a \wedge 1 = a$
    \item $a \vee 0 = a$ and $ a \wedge 0 = 0$
\end{itemize}
\end{manualprop}

\begin{defi}
An element $u$ is said to be a \textbf{complement} of an element $x \in X$ if $x \vee u = 1$ \textcolor{red}{and} $x \wedge u = 0$ 
\end{defi}

\begin{manualprop}{13}
In a distributive lattice, if an element has a complement, the complement must be unique
\end{manualprop}

\begin{defi}
A lattice $L = <X, \leq>$ is said to be \textbf{complemented} if every element $x \in X$ has a complement.
\end{defi}

\begin{defi}
A complemented and distributive lattice $L = <X, \leq>$ os also called a \textbf{Boolean lattice}
\end{defi}

\begin{namedtheorem}[Uniqueness of Boolean Lattice]
Let $B = <X, \leq>$ be a \textbf{finite} boolean lattice with associated boolean algebra $<X, \vee, \wedge, \overline{X}>$. Then there exists a finite set S such that B is isomorphic to $<\mathbb{P}(S), \subseteq>$ with associated boolean lattice $<\mathbb{P}(S), \cup, \cap, \overline{X}>$. In particular,$|X| = 2^n$, where $n = |S|$
\end{namedtheorem}


\section{\underline{Combinatorial Counting}}
\subsection{\underline{Permutations}}
\begin{manualprop}{14}
Let $X$ be a finite set with $|X| = n, n \in \mathbb{N}$, and $Y$ be a finite set with $|Y| = m, m \in \mathbb{Z^{+}}$. Then the number of all possible functions(or mappings) $f: X \xrightarrow[]{} Y$ is $m^n$
\end{manualprop}

\begin{manualcoro}{14.1}
Let $X$ be a finite set with $|X| = n, n \in \mathbb{N}$. Then the number of subsets of $X$ is $2^n$, i.e. $|\mathbb{P}(X)| = 2^n$
\end{manualcoro}


\begin{manualcoro}{14.2}
Let $X$ be a finite set with $|X| = n, n \in \mathbb{Z^{+}}$. Then the number of odd-sized subsets of $X$ is exactly equal to the number of even-sized subsets, i.e. $2^{n-1}$ in both cases
\end{manualcoro}

\begin{manualprop}{15}
The number of injective functions(mappings) from an $n-element$ set to an $m-element$ set, $m, n \in \mathbb{N}$ is exactly $m(m-1)(m-2).....(m-(n-1))$ = $\prod_{i = 0}^{n-1} (m-i)$
\end{manualprop}

\begin{defi}
A bijective function from a finite set $X$ to itself is called a permutation
\end{defi}

\begin{manualprop}{16}
The number of permutations of an $n-element$ set, $n \in \mathbb{N}$, is exactly $n(n-1)....2.1$
\end{manualprop}

\begin{defi}
$M_n = \{1, 2, 3, ...n\}$
The set of all permutations on $M_n$ is denoted by $S_n$(occasionally $\Sigma_n$, and is usually referred to as the \textbf{\textit{symmetric group on n symbols}}. $S_n = n!$
\end{defi}

\begin{manualprop}{17}
The set $S_n$ satisfies for following properties for $\pi, \sigma, \tau \in S_n$
\begin{itemize}
    \item $\pi \circ \sigma \in S_n$
    \item $(\pi \circ \sigma) \circ \tau = \pi \circ (\sigma \circ \tau)$
    \item $\pi \circ \iota = \iota \circ \pi = \pi$, where $\iota$ indicates the identity permutation 
    \item If $\pi \in S_n$, then $\pi^{-1} \in S_n$ and $\pi \circ \pi^{-1} = \pi^{-1} \circ \pi = \iota$
\end{itemize}
\end{manualprop}

\begin{defi}
Given a permutation $\sigma \in S_n$, let $M(\sigma) = \{x \in M_n: \sigma(x) \neq x\}$ i.e. the set of symbols \textbf{moved} by $\sigma$.
Similarly, let $Fix(\sigma) = \{x \in M_n: \sigma(x) = x\}$, namely, the set of symbols fixed by $\sigma$. We say that two permutations $\sigma$ and $\pi$ are disjoint if $M(\sigma) \cap M(\pi) = \phi$ 
\end{defi}

\begin{defi}
The cycle $(x_1, x_2, .... x_t)$ is the permutation which takes $x_1$ to $x_2$, $x_2$ to $x_3$, ..... $x_{t-1}$ to $x_t$, $x_t$ to $x_1$, and leaves all other symbols fixed. It is referred to as a  cycle of length $t$ or a $t-cycle$  
\end{defi}

\begin{manualprop}{18}
Any permutation $\sigma \in S_n$ can be decomposed as the product of pairwise disjoint cycles. Furthermore, this decomposition is unique up to rearranging the cycles and the cyclic order within the cycles. 
\end{manualprop}

\begin{defi}
Let $\pi \in S_n$ and let $\pi$ be decomposed as the product of disjoint cycles of lengths $n_1 \geq n_2 \geq .... \geq n_k \geq 1$. Then $n_1 + n_2 + ... + n_k = n$ and we say that the \textbf{cycle type} of $\pi$ is $(n_1, n_2, ...n_k)$. If the cycle type of $\pi$ is $(k, 1, ..., 1$, then we refer to $\pi$  as a $k-cycle$
\end{defi}

\begin{defi}
A permutation $\tau \in S_n$ is called a \textbf{transposition} provided 
\begin{itemize}
    \item $\exists i, j \in M_n, i \neq j$ such that $\tau(i) = j$ and $\tau(j) = i$ 
    \item $\forall k \in M_n, \tau(k) = k, k \neq i,j$
\end{itemize}
In other words, transposition is a 2-cycle 
\end{defi}

\begin{manualprop}{19}
Every permutation in $S_n$ can be expressed as the product(composition) of transpositions. 
\end{manualprop}



\begin{defi}
Let $\pi \in S_n$ and let $i, j \in M_n$ with $i < j$. The pair $(i,j)$ is called an \textbf{inversion} in $\pi$ if $\pi(i) > \pi(j)$
\end{defi}


\begin{manuallemma}{20.1}[\textcolor{red}{important}]
Let $(cd)$ be a transposition in $S_n$ with $c<d$. Then the number of inversions in $(cd)$ is $2(d-c-1) + 1$ i.e. an odd number of inversions.
\end{manuallemma}

\begin{manuallemma}{20.2}[\textcolor{red}{important}]
If the identity permutation is written as a product of transpositions, then the number of transpositions must be even.
\end{manuallemma}

\begin{manualprop}{20}[\textcolor{red}{important}]
    Let $\pi \in S_n$ and suppose $\pi$ is decomposed into transpositions as: $\pi = \tau_1 \cdot \tau_2 \cdot ...... \cdot \tau_a$ and $\pi = \sigma_1 \cdot \sigma_2 \cdot ...... \cdot \sigma_b$. Then $a$ and $b$ have the same \textbf{\emph{parity}} i.e. either both are odd or both are even.
\end{manualprop}

\begin{defi}
Let $\pi \in S_n$. Then $\pi$ is said to be \textbf{odd/even} provided it can be written as a product of even or odd number of transpositions resp. 
\end{defi}

\begin{manualprop}{21}
    For $n\geq 2$, the number of odd permutations is equal to the number of even permutations
\end{manualprop}

\begin{defi}
For a permutation $\pi \in S_n$, it's \textbf{sign} or \textbf{signature} is defined by $sgn(\pi) = 1$ if and only if $\pi$ is even, and $sng(\pi) = -1$ if and only if $\pi$ is odd. In particular, the signature of a transposition is $(-1)$; for a $k-cycle$ $\pi = (a_1 a_2 ... a_k)$, $sgn(\pi) = (-1)^{k-1}$ 
\end{defi}

\begin{manualtheorem}{4 (Determinant Formula)}
    If $A = [a_{ij}]$ is a square $n \times n$ matrix, then $det(A) = \sum\limits_{\sigma \in S_{n} } sgn(\sigma) a_{1\sigma_{(1)}} a_{2\sigma_{(2)}} .... a_{n\sigma_{(n)}}$
\end{manualtheorem}




\subsection{\underline{Combinations}}
\begin{defi}
    Let $n \geq k$ be non-negative integers. The \textbf{binomial coefficient} $B(n, k)$ is a function of the variables $n, k$ defined by the formula \[
    B(n, k) = \frac{n(n-1)...(n-(k-1))}{k!} =\frac{\prod_{0}^{k-1} (n-i)}{k!} = \frac{n!}{(n-k)!k!} = \binom{n}{k}
    \]
\end{defi}

\begin{defi}
    Let $X$ be a set and let $k$ be a non-negative integer. The notation $B(X, k)$ denotes the family (set) of all $k-element$ subsets of $X$.
\end{defi}

\begin{manualprop}{26}
    Let $X$ be a finite set. Then, $|B(X, k)| = B(|X|, k)$
\end{manualprop}

\begin{manualprop}{27}
    \begin{itemize}
        \item $B(n, k) = B(n, n-k)$
        \item \textbf{[Pascal's identity]} $B(n-1, k-1) + B(n-1, k) = B(n, k)$
        \item \textbf{[Binomial theorem]} For any $n \in \mathbb{N}, (1+x)^n = \sum_{k=0}^{n} B(n, k)x^k$
    \end{itemize}
\end{manualprop}

\begin{defi}
    Given $n, k_1, k_2, .. k_m \in \mathbb{Z^+}$ with $n = k_1+ k_2 + ...+ k_m$, the \textbf{multinomial coefficient} $B(n, k_1, k_2, ... k_m) = n! / (k_1 ! k_2 ! .... k_m) !$ 
\end{defi}

\begin{manualprop} {28}
    \begin{itemize}
        \item Suppose we have objects of $m$ different kinds, with $k_i$ indistinguishable objects of the $i^{th}$ kind, and $n = k_1 + k_2 + ... + k_m$, then the number of distinct arguments is given by the multinomial coefficient.
        \item \textbf{(Multinomial Theorem)} For arbitrary $x_1, x_2, ..., x_m \in \mathbb{R}$ and $n \in \mathbb{Z^+}$, we have $$(x_1 + x_2 + .. x_m) = \sum_{k_1+k_2+...k_m = n} B(n, k_1, k_2, ..., k_m) {x_1}^{k_1} {x_2}^{k_2} ...  {x_m}^{k_m}$$
    \end{itemize}
\end{manualprop}


\begin{manualprop}{29}
    \textbf{(Piegonhole Principle)} If we have $n$
boxes and we place more than $n$ objects into them, then at
least one box contains more than one object.
\end{manualprop}



\section{\underline{Finite and Infinite Sets}}


\begin{defi}
\begin{itemize}
\item The empty set $\phi$ has 0 elements.
\item If $n \in \mathbb{Z^+}$, a set $S$ has $n$ elements if there exists a bijection on $S$ onto $M_n = \{1, 2, 3, ..., n \}$
\item A set $S$ is infinite if it's not finite
\end{itemize}
\end{defi}

\begin{defi}
    A set $S$ is infinite if there exists a
proper subset $T$ of $S$ such that there is a bijection
on $S$ onto $T$. A set $S$ is finite if there exists no
proper subset $T \subseteq S$ such that there is a bijection
on $S$ onto $T$, i.e. if $S$ is not infinite.
\end{defi}

\begin{manualprop}{22}
    
\begin{itemize}
\item If a set $A$ has $m$ elements and a set $B$ has $n$ elements, with $A \cap B = \phi$, then the set $A \cup B$ has $m+n$ elements.

\item If $A$ is a set with $m$ elements and $C \subseteq A$ is a set with 1 element, then $A-C$ has $m-1$ elements
\item  If $C$ is a infinite set and $B$ is a finite set, then $C-B$ is an infinite set

\end{itemize}
\end{manualprop}

\begin{manualprop} {23}
    Suppose $S$ and $T$ are sets with $T \subseteq S$, then :-
    \begin{itemize}
        \item If $S$ is finite, then so is $T$
        \item If $T$ is infinite, then so is $S$
    \end{itemize}
\end{manualprop}

\subsection{\underline{Countable and Uncountable Sets}}
\begin{defi}
\begin{itemize}
    \item A set $S$ is \textbf{Denumerable} if there exists a bijection from S onto $\mathbb{N}$.
    \item A set S is \textbf{Countable} if it is either finite or denumerable.
    \item A set S is \textbf{uncountable} if it not countable.
    \item A set S is \textbf{Countably infinite} if it is denumerable, i.e. if it is countable but not finite. 
\end{itemize}
\end{defi}

\begin{manualprop}{24}
    Suppose $S$ and $T$ are sets with $T \subseteq S$, then:-
    \begin{itemize}
        \item If $S$ is countable, then so is $T$
        \item If $T$ is uncountable, then so is $S$
    \end{itemize}

\end{manualprop}

\begin{manualprop} {25}
    The following are equivalent :-
    \begin{itemize}
        \item S is a countable set.
        \item There exists a surjection from $\mathbb{N}$ onto $S$.
        \item There exists an injection from $S$ into $\mathbb{N}$
    \end{itemize}
\end{manualprop}

\begin{namedtheorem}[Cantor]
    If $A$ is any set, there is no surjection on $A$ onto $\mathbb{P}(A)$.
\end{namedtheorem}

\begin{manualcoro}{5.1}
    $\mathbb{P}(\mathbb{N})$ is uncountable
\end{manualcoro}

\begin{defi}
    Two sets $S$ and $T$ are said to be \textbf{equipotent} or have the \textbf{same cardinatity} if there is a bijection on $S$ onto $T$. \textcolor{red}{In view of Cantor's theorem, we can say that we have a never-ending progression of sets of increasing cardinality. }
\end{defi}

\begin{manualcoro}{5.2}
    The open interval $(0, 1)$ is uncountable.
\end{manualcoro}

\begin{manualcoro}{5.3}
    The open interval $(0, 1)$ and the set of real numbers $\mathbb{R}$ have the same cardinality.
\end{manualcoro}

\begin{manualcoro}{5.4}
    The open interval $(0, 1)$ and  $\mathbb{P}(\mathbb{N})$ have the same cardinality.
\end{manualcoro}


\section{\underline{Logic}}

\begin{defi}
    A \textbf{proposition} is a declarative
statement that is either TRUE or FALSE \textbf{(but not
both).}
\end{defi}

\begin{table}[!h]
\begin{center}
\centering
    \begin{tabular}[t]{|c|c|}
        \hline Symbol & Definition \\ \hline 
         $p$  & Assertion \\
         $\neg p$ & Negation(not) \\ 
         $p \lor q$ & Disjunction(or) \\ 
         $p \land q$ & Conjuction(and) \\
         $p \rightarrow q$ & Implication(if.. then) \\
         $p \leftrightarrow q$ & Bi-implication(iff) \\
         No standard symbol & Exclusive(xor) \\
         $q \rightarrow p$ & Converse \\
         $\neg p \rightarrow \neg q$ & Inverse \\
         $\neg q \rightarrow \neg p$ & Contra-positive \\
         $\forall$ & For all quantifier \\
         $\exists$ & Exists quantifier \\
         \hline
         
    \end{tabular}
    \caption{\label{abc}Logic Symbols and their Definition.}
\end{center}
\end{table}

\begin{defi}
    When two compound statements
always have the same truth value, they are called
\textbf{equivalent}. In other words, the rows/columns in the
corresponding truth tables have the same truth
values.
\end{defi}

\begin{defi}
    \begin{itemize}
        \item Tautology - always true
        \item Contradiction - always false
        \item Contingency - neither a Tautology or Contradiction
    \end{itemize}
\end{defi}

\begin{defi}
    The compound propositions $A$ and $B$ are called logically equivalent if $A \leftrightarrow B$ is a tautology, or $A \equiv B$
\end{defi}

\begin{defi}
    \begin{itemize}
        \item An \textbf{argument} in propositional logic is a sequence of
propositions. We start an argument with some
propositions, the \textbf{premises} or \textbf{hypotheses}. The final
proposition in the sequence is called the \textbf{conclusion}.

        \item An argument is valid if the truth of all its premises
implies that the conclusion is true. We can always use truth table to find this out
    \end{itemize}   
\end{defi}

% Please add the following required packages to your document preamble:
% \usepackage{multirow}
% \usepackage[table,xcdraw]{xcolor}
% If you use beamer only pass "xcolor=table" option, i.e. \documentclass[xcolor=table]{beamer}
\begin{center}
\begin{longtable}{|l|l|ll|}
\hline
Name                                     & Tautology                                                                                     & \multicolumn{2}{l|}{Rule}                                               \\ \hline
                                         &                                                                                               & \multicolumn{2}{l|}{{\color[HTML]{FE0000} $p \rightarrow q$}}           \\
                                         &                                                                                               & \multicolumn{2}{l|}{{\color[HTML]{FE0000} $p$}}                         \\
\multirow{-3}{*}{Modus Ponens}           & \multirow{-3}{*}{$((p \rightarrow q) \land (p)) \rightarrow q$}                               & \multicolumn{2}{l|}{{\color[HTML]{329A9D} $q$}}                         \\ \hline
                                         &                                                                                               & \multicolumn{2}{l|}{{\color[HTML]{FE0000} $p \rightarrow q$}}           \\
                                         &                                                                                               & \multicolumn{2}{l|}{{\color[HTML]{FE0000} $\neg q$}}                    \\
\multirow{-3}{*}{Modus Tollens}          & \multirow{-3}{*}{$((p \rightarrow q) \land (\neg q)) \rightarrow \neg p$}                     & \multicolumn{2}{l|}{{\color[HTML]{329A9D} $\neg p$}}                    \\ \hline
                                         &                                                                                               & \multicolumn{2}{l|}{{\color[HTML]{FE0000} $p \rightarrow q$}}           \\
                                         &                                                                                               & \multicolumn{2}{l|}{{\color[HTML]{FE0000} $q \rightarrow r$}}           \\
\multirow{-3}{*}{Hypothetical Syllogism} & \multirow{-3}{*}{$((p \rightarrow q) \land (q \rightarrow r)) \rightarrow (p \rightarrow r)$} & \multicolumn{2}{l|}{{\color[HTML]{329A9D} $p \rightarrow r$}}           \\ \hline
                                         &                                                                                               & \multicolumn{2}{l|}{{\color[HTML]{FE0000} $p \lor q$}}                  \\
                                         &                                                                                               & \multicolumn{2}{l|}{{\color[HTML]{FE0000} $\neg p$}}                    \\
\multirow{-3}{*}{Disjunctive Syllogism}  & \multirow{-3}{*}{$((p \lor q) \land (\neg p)) \rightarrow (q)$}                               & \multicolumn{2}{l|}{{\color[HTML]{329A9D} $q$}}                         \\ \hline
                                         & $p \rightarrow (p \lor q)$                                                                    & {\color[HTML]{FE0000} p}           & {\color[HTML]{FE0000} q}           \\
\multirow{-2}{*}{Addition}               & $q \rightarrow (p \lor q)$                                                                    & {\color[HTML]{329A9D} $p \lor q$}  & {\color[HTML]{329A9D} $p \lor q$}  \\ \hline
                                         & $(p \land q) \rightarrow p$                                                                   & {\color[HTML]{FE0000} $p \land q$} & {\color[HTML]{FE0000} $p \land q$} \\
\multirow{-2}{*}{Simplification}         & $(p \land q) \rightarrow p$                                                                   & {\color[HTML]{329A9D} p}           & {\color[HTML]{329A9D} q}           \\ \hline
                                         &                                                                                               & \multicolumn{2}{l|}{{\color[HTML]{FE0000} $p \land q$}}                 \\
                                         &                                                                                               & \multicolumn{2}{l|}{{\color[HTML]{FE0000} $\neg p \land r$}}            \\
\multirow{-3}{*}{Resolution}             & \multirow{-3}{*}{$((p \land q) \lor (\neg p \land r)) \rightarrow (q \land r)$}               & \multicolumn{2}{l|}{{\color[HTML]{329A9D} $q \land r$}}                 \\ \hline

\end{longtable}
\end{center}

\begin{defi}
    \textbf{Predicate logic} is concerned with
expressions of the form $P(x)$, known as
\textbf{propositional functions}, in which $P$ is a predicate or
property and $x$ is a variable. The number of variables
in a propositional function need not be one, but it
has to be finite.
\end{defi}

\begin{defi}
    When a quantifier is used on a variable
$x$ in a propositional function, that variable is said to
be \textbf{bound}. Variables which are not bound are said to \textbf{free}.
\end{defi}
Statements involving predicates and
quantifiers are logically equivalent iff they have the
same truth value no matter which predicates are
substituted into these statements and which domain
of discourse is used for the variables in these
propositional functions.
\begin{defi}
    
\end{defi}


\section{\underline{General Problem Solving Techniques}}
\begin{itemize}
\item To prove a set is a subset of another set(for example $A \subseteq B$), take some element $x \in A$ and prove $x \in B$ 
\item Induction questions regarding posets often involve considering $k=n+1$ , removing some element $a$, and constructing a new set $Y = X \setminus  \{a\}$ , applying IH on this set and then adding back $a$ to prove for $n+1$
\item  Characteristic equation comes in handy. For some subset $A \subseteq X$, 


\[
f_A(a) =\left\{
\begin{array}{ll}
      1 & a \in  A\\
      0 & otherwise \\
\end{array} 
\right.
        \]

\item Mappings are useful when talking about inclusion and exclusion. Let $A \subseteq M_n$, where $M_n = \{1, 2, ...., n\}$. Then you can create a bijection $f:A \xrightarrow[]{} V^{n}$, $V = \{0, 1\}$. $f(A) = V_A, V_A = (v_1, v_2, .... v_n)$ where $v_i = 1$ if $i \in A$, and 0 otherwise. 

\item Use Pigeonhole Principle in problems which require you to prove that there is at least one element which satisfies certain condition, or at least 2 things are same. Figuring out what the holes are and the pigeons are is half the battle 

\end{itemize}

\end{document}